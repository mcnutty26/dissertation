\documentclass{article}
\usepackage{microtype}
\author{William Seymour}
\title{Research Protocol}
\begin{document}
	\maketitle
\section{Rationale and Objectives}
The aim if the study is to see if there is correlation between two different psychological profiling techniques, one used in a gaming context and one used in an educational context. As the gamification of different everyday life becomes more pronounced, it is important that we understand how this can be best used in education. By gaining a better understanding of human-computer interaction in this way, we can more effectively educate students with these techniques.
\section{Methodology}
A questionnaire will be compiled which contains questions which categorise participants according Richard Bartle's four gamer types, and the sixteen different Myers-Briggs personality types. Some participants will be presented with software which caters to the different Bartle types and will afterwards engage in a short interview about their experiences with the software. The software components in question include such elements as leaderboards and achievement systems. The Myers-Briggs type of a participant will be used to derive their Keirsey temperament, another widely accepted analysis tool. The results will be analysed to determine whether there exists a correlation between the results from the Bartle test, and the derived Keirsey temperaments.
\section{Safety Considerations}
Both of the aforementioned personality tests are well documented, researched and widely accepted. There is no danger to participants, and all elements of the software have been created from prior research in the area, and are already widely used in gamified applications. In terms of data security, the results will be kept in secured locations while collection and analysis takes place, and destroyed at the end of the project.
\section{Dissemination of results}
Anonymised results will be made available for the purposes of evidence for the hypotheses examined by the project. As such, they will be distributed within a final report to faculty staff for marking, and may be made available for future students as an example of the execution of such a project.
\end{document}