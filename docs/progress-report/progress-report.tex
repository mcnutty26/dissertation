\documentclass[10pt,a4paper]{report}
\usepackage{graphicx}
\author{William Seymour}
\title{Dissertation Progress Review}
\date{\today \\ (Word count: 2310)}

\begin{document}
\maketitle

\section*{Introduction}
The project aims to explore how gamification and analytics can be used in the design of educational software. Gamification is a fairly young field of study, the term having only really gained popularity in late 2010, as shown in figure \ref{usage} (Google 2014). Its recent discovery has meant that far less has been done to incorporate developments on gamification into the current education system. While outside the scope of the project, one question that remains to be answered is how the engagement increases from gamification hold up when its use becomes widespread: is it only an increase relative to content that is not gamified, or can it offer an absolute increase in engagement?

In addition to the study of gamification, the project also seeks to provide an example of how analytics can be used to support lecturers and allow them to better use their time when teaching in a higher education context. This will be achieved by suggesting topics and delivery methods for guided one to one or seminar style tuition in an attempt to reinforce the progress made by students online. It will also be possible to match students who share learning styles and strength/weakness pairings in specific subject areas.

Construction of a web platform capable of the aforementioned functionality is a daunting undertaking, and the research required for it to be effective in it's aims will be substantial. Taking on a project in an exciting and relatively new field is proving to be both stimulating and rewarding.

\section*{Background Research}
Before the start of term I enrolled on a MOOC entitled gamification design (Iversity, 2014), which ran over several months and included a vast array of references and resources. In addition to this, a number of other sources of information have been identified. Of particular note are those concerned with gamification in the context of higher education, (Nicholson, 2013) (Niman, 2014) given that it would appear that comparatively less work has been done in this sector of the education system. Most of the work I have done so far focusses on gamification in a broader sense in order to gain a sound understanding of the topic as whole, to facilitate a more in depth study of the facets specific to the project over the coming months.

It is worth studying the use of gamification in other fields, such as marketing, in order to gain additional insight into why it can be so effective, and to identify techniques which could be brought across into gamification in education. Game based marketing typically employs gamification techniques such as social networking elements and leaderboards, similar to those proposed for the project. It is often found that instead of incremental gains, customer attraction and retention increases exponentially (Zichermann and Linder, 2010). 

It is also important not to forget research carried out on the \textit{flipped classroom} model which the project follows. This is where students consume lecture material in their own time outside of school or university, instead using time spent in the classroom to ask questions and complete assessments usually done at home. Such systems often incorporate online testing and analytics like those in the project. The techniques being used to generate `significant increases in student learning and achievement' (Fulton, 2012) are being adapted for use on the aforementioned web platform, with the aim of allowing lecturers to use their time with students more effectively.

While not all of the research for the project has bee completed, taking a look at the development timeline revealed that it was more appropriate to include research for the four personality types within the timeframe allotted for their development. In this way the research can better inform the programming. However, it would be foolish to just delay this research completely. Much of the broader work I have done so far has included analysis of these types, and as such, there are only a few avenues of enquiry left to persue. In structuring the research work in this way it is hoped that the later analysis can be completed swiftly.

\begin{figure}
	\includegraphics{../img/usage-graph.png}
	\caption{Usage of the term `gamification' by month as a proportion of the total number of google searches for gamification, from 2009 to the present day (Google 2014)}
	\label{usage}
\end{figure}

\section*{Technical Content}
Over the past eight weeks, work has been progressing largely according to the timeline set forth in the specification. The core of the platform has been completed, including the following high level features. Figure \ref{db} contains an overview of the database which serves the project. It uses MySQL, and is currently hosted by the University's Computing Society. Database development uses phpmyadmin as installed on the Computing Society's server, which was chosen over the mysql command line interface (CLI) because it was easier to use without any prior experience, and the extra features that the CLI offers are either not required or can be exposed via running SQL commands through phpmyadmin. An overview of the classes used in the software is given in figure \ref{classes}.  PHP development is done using JetBrains PHP Storm, a powerful development environment used extensively in industry. Having already had some experience using PHP Storm, it was the logical choice given how its features allow for quicker and less error prone development. To date, the following features have been completed:

\subsection*{User Accounts}
In addition to an ID number and a username, those using the system also have persistent statistics stored and associated with their profile. This currently comprises information on their Bartle personality type, as derived from answers given to questions from course content, and a score aggregate. These values are used to generate a mask containing the users dominant personality types. Profile data is updated at the end of each module based on the answers given. A time weighted average is taken of the new scores and existing data according to a time decay constant defined in the user class. This value determines how quickly data decays in importance, or inversely, how much more relevant new data is than old data.

\subsection*{Content Creation}
There is currently basic functionality allowing the creation and deletion of content and structural database elements. The logic for the creation page is mainly concerned with selecting and inserting database elements correctly. If there is enough time towards the end of the project, these tools could be improved, but they are currently sufficient to demonstrate the other features of the software.

\subsection*{Content Delivery}
The content delivery pages present users with questions from the module they selected. Once a question has been answered, the response chosen is logged in the database, and the user is presented with the next question in the module for which they have not given an answer. If that was the only remaining unanswered question, they are instead directed to the module review page. This page shows the users persistent statistics, as well as their performance on the last module they took.

\begin{figure}
	\includegraphics[width=\textwidth]{../img/database.png}
	\caption{Diagram of the current project database. Arrows denote foreign key relationships. Foreign key fields are prefixed with `F\_' and have the field they reference appended after the `-$\rangle$'.}
	\label{db}
\end{figure}

\begin{figure}
	\includegraphics[width=\textwidth]{../img/classes.png}
	\caption{An overview of the classes used in the software.}
	\label{classes}
\end{figure}

\section*{Progress and Next Steps}
I realised fairly early into the term that the timetable I had submitted as part of the specification was a bit more demanding than was practical. While it achieved its objective in that it encouraged me to work hard from the start of the project, I am currently a week behind schedule after not having enough time available to complete all the work I had timetabled. In terms of the core of the application - the components required to facilitate the gamification features - development has finished, and these features are ready to use. This will allow me to begin researching and developing the areas that are linked to each Bartle personality type. A brief explanation of these personality types is given in appendix I. In addition to development, I have also been researching the wider usage and implementation of gamification. A selection of these resources are referenced in the background research section.

After the four Bartle areas have been covered, the product will be thoroughly tested for bugs, and user feedback sought from a small focus group. In the focus group, a small number of people from the target demographic (undergraduate students) will be given a number of sample modules generated from computer science courses and asked to complete them. This data will be analysed in terms of assessing how accurately the software tracks a students user type, and how effective the tools targeted at each user were. These can be partially verified against a widely used online Bartle test available at \textit{gamerdna.com}. Any major bugs will be fixed during this time, and any minor ones logged to be worked on should there be spare time towards the end of the project. Finally, the presentation will be delivered and the report compiled between the end of term two and the start of term three.

\section*{Appraisal}
There was an error in the original timetable submitted in that the time set aside for writing this report was scheduled for after the due date. This, in addition to the scheduling issue mentioned above, has been rectified in a revised timetable contained in appendix II. The timeline is the only portion of the original specification that has been altered. On reflection, I am glad that I took the time to fully assess the data that would need to be stored in the database, as once the database structure is determined it is very costly to change. As such, being able to leave it alone during the rest of development has saved me valuable time. The scope of the project is large, and in order to ensure its completion I have had to distil the core elements down to their simplest form in order to develop and integrate them in the allotted time. To this end, the project as it stands appears quite bare bones, but I feel that this is acceptable given that the project is more of a gamification/split classroom prototype than an exercise in rigorous software development. As mentioned before, any spare time towards the end of the project can be used to add styling and other visual elements.

\section*{Ethics}
As described in appendix I, ethical consent will have to be obtained for the user feedback testing planned for week five of term two. This approval still needs to be sought. However, as the activities proposed are eligible for devolved consent this will not take long, and will be completed at the beginning of term two.

\section*{Project Management}
In terms of project management, I have been following a variant of the incremental build model, as identified in appendix I. This has been working well, and has allowed me to timetable software development more accurately than if a more popular agile methodology had been used. As mentioned in appendix I, the typical agile project has a basic version of the functionality developed before incremental improvements are added in each subsequent sprint. 

The use of GitHub as a source and document control tool has been extremely useful, and it's use as a backup location has already proved invaluable once after an equipment failure. Being able to easily access and synchronise project files across multiple locations and physical devices has also been a great boon for productivity. As a less formal form of project management, I have also been meeting with my project supervisor as required. These meetings have been used to ensure that the project is proceeding in a timely manner and that any documents that need to be submitted are of a sufficient standard. A log of all meetings to date can be found in appendix III.

\section*{Appendices}

\subsection*{Appendix I - Original Project Specification}
\input{../spec/Spec-include}

\subsection*{Appendix II - Revised Timetable}
\begin{tabular}{|c|c|c|}
	\hline Term/Week & Tasks & Changed from v1 \\ 
	\hline 1/1 & Specification & No \\ 
	\hline 1/2 & Specification & No \\ 
	\hline 1/3 & Core App, Environment & No \\ 
	\hline 1/4 & Core App, Content Delivery & No \\
	\hline 1/5 & Content Delivery & Yes \\
	\hline 1/6 & Analytics & No \\ 
	\hline 1/7 & Analytics & Yes \\
	\hline 1/8 & Progress Report & Yes \\
	\hline 1/9 & Socialiser features & Yes \\
	\hline 1/10 & Socialiser/Achiever features & Yes \\
	\hline Christmas Holiday & Buffer zone (catchup) & No \\
	\hline 2/1 & Achiever Features & Yes \\ 
	\hline 2/2 & Killer Features & Yes \\ 
	\hline 2/3 & Killer/Explorer features & Yes \\ 
	\hline 2/4 & Explorer Features & Yes \\ 
	\hline 2/5 & Product testing and feedback sessions & Yes \\ 
	\hline 2/6 & Feedback analysis & Yes \\ 
	\hline 2/7 & Report and presentation writing & Yes \\ 
	\hline 2/8 & Report and presentation writing & Yes \\ 
	\hline 2/9 & Report and presentation & Yes \\ 
	\hline 2/10 & Report and presentation & Yes \\
	\hline Easter Holiday & Report & No \\
	\hline 
\end{tabular}

\subsection*{Appendix III - Meeting Log}
\begin{tabular}{|c|c|c|}
	\hline Date & Format \\ 
	\hline Thu 2nd Oct 2014 & Face to face \\ 
	\hline Mon 6th Oct & Face to face \\
	\hline Tue 7th Oct 2014 & Email \\ 
	\hline Fri 31st Oct 2014 & Face to face \\ 
	\hline Wed 12th Nov 2014 & Face to face \\
	\hline Tuesday 13:30 & Face to face \\ 
	\hline Tue 18th Nov 2014 & Face to face \\ 
	\hline 
	\end{tabular} 

\section*{Bibliography}
Google, 2014. \textit{Google Trends- Web Search interest: gamification - Worldwide, 2009 - 2014}. [image online] Available at: 

\noindent $\langle$http://www.google.com/trends/explore\#q=gamification\&date=1\%2F2009\%2072m\&cmpt=q$\rangle$ [Accessed 09/11/2014].
\newline

\noindent Iversity, 2014. \textit{Gamification Design Online Course}. [online] Available at: 

\noindent $\langle$https://iversity.org/en/courses/gamification-design-online-course$\rangle$ [Accessed 18/03/2014].
\newline

\noindent Nicholson, S., 2013. \textit{Exploring Gamification Techniques for Classroom Management}. [online] Available at: $\langle$http://scottnicholson.com/pubs/gamificationtechniquesclassroom.pdf$\rangle$ [Accessed 21/10/2014].
\newline

\noindent Niman N., 2014. \textit{The Gamification of Higher Education}. [online] Available at: $\langle$http://0-www.palgraveconnect.com.pugwash.lib.warwick.ac.uk/pc/doifinder/10.1057/9781137331465$\rangle$ [Accessed: 13 November 2014].
\newline

\noindent Zichermann, G. and Linder, J., 2010. \textit{Game-Based Marketing}. New Jersey: Wiley.
\newline

\noindent Fulton, K., 2012. \textit{Upside Down and Inside Out: Flip Your Classroom to Improve Student Learning}. [online] Available at $\langle$http://files.eric.ed.gov/fulltext/EJ982840.pdf$\rangle$ [Accessed: 13 November 2014].
\end{document}