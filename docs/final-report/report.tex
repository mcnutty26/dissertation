\documentclass{report}
\title{CS310 Final Report \\ The Use of Gamification and Analytics in Higher Education}
\author{William Seymour, Third Year Computer Science \\ Supervisor: Mike Joy}
\date{\today}
\linespread{1.3}

\usepackage{cite}

\begin{document}
\maketitle
\clearpage
\tableofcontents
\section{Abstract}

\section{Introduction}

\section{Definitions}
It is important to define some of the key terms that will be used in this report. Terms like gamification and analytics are often bandied around by lots of different parties who all mean similar but subtly different things. Indeed, the way in which these terms have been used thus far have been somewhat unclear. For the remainder of the document, the undermentioned should be referred to as a definitive explanation of what is meant by various technical terms.

\subsection{Gamification}
A commonly accepted definition of gamification is that it is the use of game design elements in non-game contexts \cite{deterding2011game}. This is a good in that it makes clear the distinction between games and gamified activities, but what exactly counts as either is often unclear. This is difficult as it is near impossible to pin down where one stops and the other begins. This is likely to increase as game elements are further incorporated into everyday activities, and seems to suggest that rigorously defining what is and is not a game is of little value. 

Game elements also need to be outlined in a little more depth. For the purposes of this project, it will be taken to mean those elements of game systems which compel users to play because they are engaging or fun. Alternatively put, the focus is on the mechanics that make games interesting and keep users coming back as opposed to graphical techniques or the platforms they reside on. Further definition, as before, carries the risk of splitting hairs, and does not add much to the discussion.

\subsection{Analytics}

\section{Conclusion}

\bibliography{references}
\bibliographystyle{plain}
\end{document}